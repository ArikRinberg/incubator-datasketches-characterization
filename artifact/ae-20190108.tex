% LaTeX template for Artifact Evaluation V20180713
%
% Prepared by 
% * Grigori Fursin (cTuning foundation, France and dividiti, UK) 2014-2019
% * Bruce Childers (University of Pittsburgh, USA) 2014
%
% See example of this Artifact Appendix in
%  * SC'17 paper: https://dl.acm.org/citation.cfm?id=3126948
%  * CGO'17 paper: https://www.cl.cam.ac.uk/~sa614/papers/Software-Prefetching-CGO2017.pdf
%  * ACM ReQuEST-ASPLOS'18 paper: https://dl.acm.org/citation.cfm?doid=3229762.3229763
%
% (C)opyright 2014-2019
%
% CC BY 4.0 license
%

\documentclass{sigplanconf}

\usepackage{hyperref}

\begin{document}

\special{papersize=8.5in,11in}

%%%%%%%%%%%%%%%%%%%%%%%%%%%%%%%%%%%%%%%%%%%%%%%%%%%%
% When adding this appendix to your paper, 
% please remove above part
%%%%%%%%%%%%%%%%%%%%%%%%%%%%%%%%%%%%%%%%%%%%%%%%%%%%

\appendix
\section{Artifact Appendix}

%%%%%%%%%%%%%%%%%%%%%%%%%%%%%%%%%%%%%%%%%%%%%%%%%%%%%%%%%%%%%%%%%%%%%
\subsection{Abstract}

The artifact contains all the JARs of version 0.12 of the DataSketches
library, before it moved into Apache (Incubating), as well as configurations
and shell scripts to run our tests.. It can support the results found in
Section XXX of our PPoPP'2020 paper Fast Concurrent Data Sketches. To
validate the results, run the test scripts and check the results piped
in the according text output files.

\subsection{Artifact check-list (meta-information)}

{\em Obligatory. Use just a few informal keywords in all fields applicable to your artifacts
and remove the rest. This information is needed to find appropriate reviewers and gradually 
unify artifact meta information in Digital Libraries.}

{\small
\begin{itemize}
  \item {\bf Algorithm: HLL $\Theta$ Sketch}
  \item {\bf Program: Java code}
  \item {\bf Compilation: Each package is compiled using mvn clean PACKAGE}
  \item {\bf Binary: Java executables}
  \item {\bf Run-time environment: Java}
  \item {\bf Hardware: Ubuntu on 12 core server and 32 core server with hyperthreading disabled}
  \item {\bf Execution: }
  \item {\bf Metrics: Throughput}
  \item {\bf Output: Runtime throughputs}
  \item {\bf Experiments: }
  \item {\bf How much disk space required (approximately)?: Not much.}
  \item {\bf How much time is needed to prepare workflow (approximately)?: Using precomipled packages, none.}
  \item {\bf How much time is needed to complete experiments (approximately)?: 10 hours}
  \item {\bf Publicly available?: Yes.}
  \item {\bf Code licenses (if publicly available)?: Apache License 2.0}
  \item {\bf Data licenses (if publicly available)?: }
  \item {\bf Workflow framework used?: }
  \item {\bf Archived (provide DOI)?: }
\end{itemize}

%%%%%%%%%%%%%%%%%%%%%%%%%%%%%%%%%%%%%%%%%%%%%%%%%%%%%%%%%%%%%%%%%%%%%
\subsection{Description}

\subsubsection{How delivered}

{\em Obligatory}
The Apache DataSketches (Incubating) library is an open source project
under Apache License 2.0, and is hosted with code, API specifications,
build instructions, and design documentations on Github.

\subsubsection{Hardware dependencies}
Our tests require a 12 core server and 32 core machine with hyper-threading disabled

\subsubsection{Software dependencies}
The Apache DataSketches (Incubating) library has been tested on Ubuntu 12.04/14.04,
and is expected to run correctly under other Linux distributions.

\subsubsection{Data sets}

%%%%%%%%%%%%%%%%%%%%%%%%%%%%%%%%%%%%%%%%%%%%%%%%%%%%%%%%%%%%%%%%%%%%%
\subsection{Installation}

{\em Obligatory}

We have provided the necessary JAR files for recreating our experiment:
\begin{itemize}
  \item sketches-core-0.12.1-SNAPSHOT.jar
  \item memory-0.12.1.jar
  \item characterization-0.1.0-SNAPSHOT.jar
\end{itemize}

Alternatively, follow the build instructions on Apache DataSketches (Incubating) apache
page (https://datasketches.apache.org/), in order to building the above mentioned
JAR files, now called:
\begin{itemize}
  \item incubator-datasketches-java (https://github.com/apache/incubator-datasketches-java)
  \item incubator-datasketches-memory (https://github.com/apache/incubator-datasketches-memory)
  \item incubator-datasketches-characterization (https://github.com/apache/incubator-datasketches-characterization)
\end{itemize}
\noindent Maven is required to build these JAR files. The version number of incubator-datasketches-java
and incubator-datasketches-memory must comply with the version numbers required by incubator-datasketches-characterization.


%%%%%%%%%%%%%%%%%%%%%%%%%%%%%%%%%%%%%%%%%%%%%%%%%%%%%%%%%%%%%%%%%%%%%
\subsection{Experiment workflow}
For convenience, we provide the JAR files required and the configurations
used to run our tests.

\begin{itemize}
  \item Clone the repository to the local machine:

  \hrulefill

  \$ git clone https://github.com/ArikRinberg/incubator-datasketches-characterization

  \$ cd incubator-datasketches-characterization/artifact

  \hrulefill

  \item Next, run the tests:

  \hrulefill

  \$ java -cp "./*" com.yahoo.sketches.characterization.Job \${FigureXX}JobTest.conf

  \hrulefill

  \noindent where FIGURE-XX is the figure that you wish to recreate. The results per
  experiment are in a file called \${FigureXX}Profile*.txt. 

\end{itemize}

%%%%%%%%%%%%%%%%%%%%%%%%%%%%%%%%%%%%%%%%%%%%%%%%%%%%%%%%%%%%%%%%%%%%%
\subsection{Evaluation and expected result}

For Figures X,Y and Z, the expected results are runtime throughput in nanoseconds
per update. These figures show updates per second, therefore a conversion is needed;
if the result is $x$, then the data-point is $1000/x$. For Figures A and B, the
expected results are accuracy.

{\em Obligatory}

%%%%%%%%%%%%%%%%%%%%%%%%%%%%%%%%%%%%%%%%%%%%%%%%%%%%%%%%%%%%%%%%%%%%%
\subsection{Experiment customization}

%%%%%%%%%%%%%%%%%%%%%%%%%%%%%%%%%%%%%%%%%%%%%%%%%%%%%%%%%%%%%%%%%%%%%
\subsection{Notes}

%%%%%%%%%%%%%%%%%%%%%%%%%%%%%%%%%%%%%%%%%%%%%%%%%%%%%%%%%%%%%%%%%%%%%
\subsection{Methodology}

Submission, reviewing and badging methodology:

\begin{itemize}
  \item \url{http://cTuning.org/ae/submission-20190109.html}
  \item \url{http://cTuning.org/ae/reviewing-20190109.html}
  \item \url{https://www.acm.org/publications/policies/artifact-review-badging}
\end{itemize}

%%%%%%%%%%%%%%%%%%%%%%%%%%%%%%%%%%%%%%%%%%%%%%%%%%%%
% When adding this appendix to your paper, 
% please remove below part
%%%%%%%%%%%%%%%%%%%%%%%%%%%%%%%%%%%%%%%%%%%%%%%%%%%%

\end{document}
